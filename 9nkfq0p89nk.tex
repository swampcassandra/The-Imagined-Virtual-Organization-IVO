.\section{What and Why is an IVO? The Explanation.}
\newthought{An Imagined Virtual Organization.} It came from the right side of my brain/mind, somewhere. It emerged after actions from conventional mind spaces were not producing results. Too much time trying to make ideas that occurred to me fit frames of assumptions I had made about my potential audience. I was hoping to for a moment attract their attention   methods that were conventional in one of the following settings.


\par \newthought{A report, like a case study, that might be familiar} to doctors and nurses, at least the older or old school persons. Practical but scholarly with references. Not very exciting or engaging but some might read it if they were convinced it would save lives, their own or patients, or keep them out of court for malpractice. \\Such reports are events. They have emotional significance but are not assigned importance in descriptions that seek a false, materialistic, rationalist view. Quite circularly such reports are designed consciously or outside awareness to specifically exclude those events from observation, description, and inclusion of the ecology of cognition.
\pagebreak


\par\newthought{An article in a journal.} Could it find a place in a psychiatry or psychoanalytic or anthropology journal.
I would have to make it politically correct and scientifically  plausible.  Very few people have a similar background or life experience and most of the time I have been a bit of a maverick and have indulged in Interdisciplinary Matters. True believers in each of these areas regard that as going over to the enemy, letting down the side. A social peril of interdisciplinary work, or clinical focus. 


\par\newthought{I an in a hurry}  Some of what I have to say is quite urgent.
Daily life is becoming a lot worse. Am I making that up? My event findings are discounted as anecdotal or at best stories. I say changes in medical care are killing more people. The public has remarkable gaps in their knowledge of medical care.  My message unsettles my colleagues and scares the part of the public that has not yet been seriously ill or had a near death experience. Those who have such experience sometimes accept I may be speaking some truth but not about \textbf{their} doctor, or hospital or their congressperson who is an exceptional caring person.  Like older military veterans, I can pick them out, there is evidence in their habitual approach to life, of some wear and tear, they have \emph{seen the elephant.} \\Those who have not yet seen the elephant do not have time to listen to me as they are too distracted and have to quickly feed their addictions to the distractions. Time for reflection and human interaction is lost in daily life. People act increasingly like consumption robots feeding on things or information in a increasingly needy interaction. 

\par\newthought{I need some support, HELP!}
Would money help. Should my efforts turn a profit? Everywhere are start-ups. But then one must explain this and acquire a business plan. I assert that business plans as currently structured are part of the problem. One not only needs a clinical or thinking team but a lawyer, an accountant and a flak catcher and maybe a body-guard. This inhibits thinking freely. I feel Sad.

\par\newthought{A Non-Profit, but are NPO's not an indirect appeal for money?}
I have been on Boards, they have many problems establishing their status, long reports to the IRS. Can one trust the IRS?  Are you kidding me? In moving from initial ideas to plans of action, reality, or apparent reality sneaks in like a fox in the hen house. The barriers to entry seem insurmountable. Ideas wilt without adequate resources and in the face of "managing by inconvenience" 
\pagebreak
\par\newthought{Several years of leaving this undefined} while pursuing an interest in medical error led to the aging of my older colleague, my moving a lot, a younger associate having development paths of his own, and my community oriented colleague forming her own organization. My understanding widened through throwing out ides and gaining others perspectives. I miss the association and stimulation. Here I am alone still thinking and profiting tremendously from what I learned from them but there is no clear path forward.

\par\newthought{Well IMAGINE.}If I could do exactly what I want with my mental/brain productions what would that be? A VIRTUAL thing, requiring no home base, no bricks and mortar, no business plan. Structure is needed. Do not get carried away. Much of the problem as I see it is how hard it is to face reality. Reality is what interferes with simple solutions in complex situations. Avoidance of reality starts with perhaps harmless illusions, Soon they are defended. We seek confirmation, and validation from others. 
We are disturbed by new events not fitting the frame, template, jig that drives our cognition and our conditioned, out of awareness, skilled behavior. \marginnote{{\textcolor{blue}These thoughts are closely related to the arguments presented in Matthew Crawford's recent interview, copy provided by Andrea.}}

%